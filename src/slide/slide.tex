\documentclass{beamer}
\usepackage[utf8]{inputenc}
\usepackage[spanish]{babel}
\usepackage{graphicx}

%%%%%%%%%%%%%%%%%%%%%%%%%%%%%%%%%%%%%%%%%%%%%%%%%%%%%%%%%%%%%%%%%%%%%%%%%%%%%%%%%%%%%%%%%%%%

\newtheorem{definicion}{Definición}
\newtheorem{ejemplo}{Ejemplo}

%%%%%%%%%%%%%%%%%%%%%%%%%%%%%%%%%%%%%%%%%%%%%%%%%%%%%%%%%%%%%%%%%%%%%%%%%%%%%%%%%%%%%%%%%%%%

\usetheme{classic}
\usecolortheme[RGB={125,60,125}]{structure}

%%%%%%%%%%%%%%%%%%%%%%%%%%%%%%%%%%%%%%%%%%%%%%%%%%%%%%%%%%%%%%%%%%%%%%%%%%%%%%%%%%%%%%%%%%%%

\title[Integración]{Método del trapecio.}
\author[Grupo-2D]{Miriam Martín Jacinto.\\Tiffany López Nicholson.\\Sergio Vega García.}
\date[01/05/2013]{1 de mayo de 2013.}

%%%%%%%%%%%%%%%%%%%%%%%%%%%%%%%%%%%%%%%%%%%%%%%%%%%%%%%%%%%%%%%%%%%%%%%%%%%%%%%%%%%%%%%%%%%%

\begin{document}
%%%%%%%%%%%%%%%%%%%%%%%%%%%%%%%%%%%%%%%%%%%%%%%%%%%%%%%%%%%%%%%%%%%%%%%%%%%%%%%%%%%%%%%%%%%%
  \begin{frame}
    \begin{figure}[lt]
      	\includegraphics[width=0.2\textwidth]{ull.eps}
      	\hspace{5.5cm}
		\includegraphics[width=0.2\textwidth]{fmatesc.eps}
    \end{figure}
    \titlepage
  \end{frame}
%%%%%%%%%%%%%%%%%%%%%%%%%%%%%%%%%%%%%%%%%%%%%%%%%%%%%%%%%%%%%%%%%%%%%%%%%%%%%%%%%%%%%%%%%%%%
  \begin{frame}
    \frametitle{Índice.}
    \tableofcontents[pausesections]
  \end{frame}
%%%%%%%%%%%%%%%%%%%%%%%%%%%%%%%%%%%%%%%%%%%%%%%%%%%%%%%%%%%%%%%%%%%%%%%%%%%%%%%%%%%%%%%%%%%%
  \section{Integración.}
  \begin{frame}
    \frametitle{Integración.}
      \begin{definicion}
	Una \textbf{integral} es una generalización de la suma de infinitos sumandos, infinitamente pequeños.
      \end{definicion}
	
      \begin{definicion}
	Dada una función f(x) y un intervalo [a,b], la \textbf{integral definida} es igual al área limitada entre la gráfica de f(x), el eje de abscisas, y las rectas verticales x = a y x = b.\\
	$\int_{a}^{b} f(x) dx$, continua en el intervalo $[a, b].$
      \end{definicion}
  \end{frame}
%%%%%%%%%%%%%%%%%%%%%%%%%%%%%%%%%%%%%%%%%%%%%%%%%%%%%%%%%%%%%%%%%%%%%%%%%%%%%%%%%%%%%%%%%%%%
  \section{Regla del trapecio.}
  \begin{frame}
    \begin{definicion}
      La \textbf{regla del trapecio} es un método de integración numérica que se basa en aproximar el valor de la integral definida de $f(x)$ por el de la función lineal que pasa a través de ésta, formándose una figura: un trapecio. Para obtener esta aproximación, debemos calcular el área de los trapecios.
    \end{definicion}
  \end{frame}
%%%%%%%%%%%%%%%%%%%%%%%%%%%%%%%%%%%%%%%%%%%%%%%%%%%%%%%%%%%%%%%%%%%%%%%%%%%%%%%%%%%%%%%%%%%%
  \section{Nuestro caso.}
  \begin{frame}
    \begin{block}{La integral.}
      En esta exposición se mostrará la siguiente integral utilizando la \textbf{regla del trapecio} y nuestro programa $Python$.\\
      \begin{center}
	$\int_{1}^{6}\frac{1}{1+e^x}dx$, en el intervalo $[1,6]$
      \end{center}
    \end{block}

    \begin{block}{¿Qué vamos a hacer?}
      Crearemos un algoritmo en \textbf{$Python$} para hacer varias pruebas, dependiendo de las partciones que haremos, utilizando la regla del trapecio.
    \end{block}
  \end{frame}
%%%%%%%%%%%%%%%%%%%%%%%%%%%%%%%%%%%%%%%%%%%%%%%%%%%%%%%%%%%%%%%%%%%%%%%%%%%%%%%%%%%%%%%%%%%%

%%%%%%%%%%%%%%%%%%%%%%%%%%%%%%%%%%%%%%%%%%%%%%%%%%%%%%%%%%%%%%%%%%%%%%%%%%%%%%%%%%%%%%%%%%%%

%%%%%%%%%%%%%%%%%%%%%%%%%%%%%%%%%%%%%%%%%%%%%%%%%%%%%%%%%%%%%%%%%%%%%%%%%%%%%%%%%%%%%%%%%%%%
  %\section{Bibliografía.}
  %\begin{thebibliography}{00}
    %\bibitem{}
    
  %\end{thebibliography}

%%%%%%%%%%%%%%%%%%%%%%%%%%%%%%%%%%%%%%%%%%%%%%%%%%%%%%%%%%%%%%%%%%%%%%%%%%%%%%%%%%%%%%%%%%%%

\end{document}

%%%%%%%%%%%%%%%%%%%%%%%%%%%%%%%%%%%%%%%%%%%%%%%%%%%%%%%%%%%%%%%%%%%%%%%%%%%%%%%%%%%%%%%%%%%%
