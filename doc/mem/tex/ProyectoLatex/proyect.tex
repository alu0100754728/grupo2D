\documentclass{article}
\usepackage[utf8]{inputenc}
\usepackage[spanish]{babel}
%\usepackage{graphicx}

\begin{document}
%Portada
  %\includegraphics[width=0.5\textwidth]{ull.png}
  \title{Integración Del Trapecio} %[En el intervalo [1, 6] de una función]
  \author{Tiffany López Nicholson \\ Miriam Martín Jacinto \\ Sergio Vega García}
  \date{\today}
  \maketitle

  \begin{abstract}
    \begin{center}
       A continuación se presentará como se ha implementado con python un algoritmo capaz de resolver la integral definida $\int_{1}^{6} \frac{1}{1+e^x}$
    \end{center}
  \end{abstract}
  \pagebreak

%%%%%%%%%%%%%%%%%%%%%

  \tableofcontents
  \pagebreak
  
%%%%%%%%%%%%%%%%%%%%%

  \section{Integración Del Trapecio}
    La regla del trapecio es un método de integración numérica. Esta regla se basa en aproximar el valor de la integral de f(x) por el de la función lineal que pasa a través  de ésta es igual al área del trapecio que se forma bajo la gráfica de la función lineal.
    \subsection{Items y enumeraciones.}
      \begin{itemize}
        \item Primero.
	\item Segundo.
      \end{itemize}
      \begin{enumerate}
        \item Tercero.
	\item Cuarto.
      \end{enumerate}
   \pagebreak

\end{document}
