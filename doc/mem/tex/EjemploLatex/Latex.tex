\documentclass{article}
\usepackage[utf8]{inputenc}
\usepackage[spanish]{babel}
\usepackage{graphicx}

\begin{document}
  \title{Ejemplo de \LaTeX.}
  \author{Tiffany López Nicholson.}
  \date{\today}
  \maketitle

  \begin{abstract}
    \begin{center}
      Esto es un resumen.
    \end{center}
  \end{abstract}

  \section{Primera sección.}
    Puedo escribir aquí. Ahora quiero un pie de página\footnote{Aquí lo tengo.}.
    \subsection{Items y enumeraciones.}
      \begin{itemize}
        \item Primero.
	\item Segundo.
      \end{itemize}
      \begin{enumerate}
        \item Tercero.
	\item Cuarto.
      \end{enumerate}
   \pagebreak
      
    \subsection{Tablas e imágenes.}
      \begin{figure}
	\begin{center}
	\includegraphics[width=0.5\textwidth]{imagen1.eps}
	\caption{Esto es una imagen.}
	\end{center}
      \end{figure}

      \begin{table}
	\begin{center}
	\begin{tabular}{|l|c|r|}
	  \hline
	  {\bf Izquierda} & {\bf Centro} & {\bf Derecha} \\ \hline
	  1 & 2 & 3 \\ \hline
	\end{tabular}
	\caption{Esto es una tabla.}
	\end{center}
      \end{table}

    \section{Fómulas y más cositas...}
      Mientras escribo, aparece una fórmula: $a^2+b^2+2ab = (a+b)^2$.\\
      Y ahora centadro:
	$$a^2+b^2+2ab = (a+b)^2$$
      Usemos fracciones, raíces y números extraños:
	$$x=\frac{-b+\sqrt[2]{b^2-4ac}}{2a}; x=\pi$$
      Para usar comillas ''HELLO WORLD!''. \~Nos gusta mucho la \~n. ESPA\~NA.
  \begin{thebibliography}{99}
    \bibitem[1]{Man}
      Manual de \LaTeX.
      \emph{Coromoto.} Profesora de Técnicas Experimentales.
  \end{thebibliography}
\end{document}